\documentclass[a4paper, twocolumn]{article}

\author{14231016 马琛骁}
\title{San Fransisco Crime Report}

\usepackage{xeCJK}
\setCJKmainfont{Songti SC}
\usepackage{indentfirst}

\usepackage{listings}

\usepackage[superscript]{cite}
\makeatletter % changes the catcode of @ to 11
\renewcommand\@citess[1]{\textsuperscript{[#1]}}
\makeatother % changes the catcode of @ back to 12

\usepackage{amsmath}

\renewcommand\tablename{表}
\renewcommand\figurename{图}

\begin{document}

\maketitle

\section{数据概览}

训练数据共有 878049 条,每条有 9 个属性,分别统计这些属性值的可能选项如表\ref{tab:train}所示。

\begin{table}[h]
    \centering
    \begin{tabular}{*{5}{r}}
        \hline
        Total& Date& Cate& Dspt& DOW \\
        \hline
        878049& 389257& 39& 879& 7 \\
        1& 2& 22514& 999& 125436 \\
        \hline
        \hline
        Pd& Res& Add& X& Y \\
        \hline
        10& 17& 23228& 34243& 34243 \\
        87805& 51650& 38& 26& 26 \\
        \hline
    \end{tabular}
    \caption{训练集属性}
    \label{tab:train}
\end{table}

测试数据共有 65499 条,每条有 6 个属性,分别统计这些属性值的可能选项如表\ref{tab:test}所示。
其中Category即模型预测的目标。

\begin{table*}[h]
    \centering
    \begin{tabular}{*{7}{r}}
        \hline
        Total& Date& DOW& Pd& Add& X& Y \\
        \hline
        65499& 28495& 7& 10& 12124& 13925& 13925 \\
        1& 2& 9357& 6550& 5& 5& 5 \\
        \hline
    \end{tabular}
    \caption{测试集属性}
    \label{tab:test}
\end{table*}

统计各个属性在训练集和测试集的重合度,结果如图\ref{fig:overlap}。

\begin{figure}[!h]
    \centering
    \includegraphics[width=0.5\textwidth]{overlap.png}
    \caption{属性重合度}
    \label{fig:overlap}
\end{figure}

注意到测试集和训练集的日期完全没有重合,
因此考虑将日期分割为“小时”,“日”,“月”,“年”四个属性,
经过统计,此时训练集和测试集完全重合。

\section{模型简介}

\subsection{逻辑回归}

由于需要预测的变量是分类的,本文首先尝试了最简单的逻辑回归模型。

最基本的逻辑回归模型假设某一随机事件发生的对数几率是特征的线性函数。
对数几率即随机事件发生的概率与不发生的概率之比的对数。

\begin{equation}
    \mathrm{log}\frac{P(Y=1|x)}{P(Y=0|x)}=w\cdot x
\end{equation}

学习逻辑回归模型的过程即估计模型参数$w$的过程,估计参数可以使用极大似然法。

\begin{equation}
    w=\operatornamewithlimits{argmax}_wP
\end{equation}
\begin{equation}
    P=\prod_{i=1}^N \left( P(Y=1|x_i)^{y_i} P(Y=0|x_i)^{1-y_i} \right)
\end{equation}

对于结果有不止两种可能的随机事件,可以将以上逻辑回归模型推广到多分类,
推广的方法有One-versus-Rest和Multinomial两种。
One-versus-Rest认为出现第$1,2,...,K-1$种结果的概率与出现第$K$种结果的概率之比的对数是特征的线性函数。

\begin{equation}
    \mathrm{log}\frac{P(Y=k|x)}{P(Y=K|x)}=w_k\cdot x
\end{equation}

Scikit Learn框架提供了两种多分类回归模型,
提供了liblinear和newton-cg等多种求解算法和L1,L2两种正则项。
本文进行了多组对照实验对比其性能,结果如表\ref{tab:logsitic}所示,
各个模型、算法即正则项之间并无显著差别。

\begin{table*}[h]
    \centering
    \begin{tabular}{*{7}{r}}
        \hline
        模型& 算法& 正则项& 训练集loss& 测试集loss \\
        \hline
        OvR& liblinear& L2& 2.655& 2.656& \\
        OvR& liblinear& L1& 2.662& 2.662& \\
        OvR& newton-cg& L2& 2.641& 2.642& \\
        Multinomial& newton-cg& L2& 2.652& & \\
        \hline
    \end{tabular}
    \caption{逻辑回归分组实验}
    \label{tab:logsitic}
\end{table*}

\subsection{决策树}

决策树是一种机器学习模型,
它通过对训练数据进行分析,得到一个多次选择的分类器,
每一次根据样本的某个或某些属性进行决策,最终得到样本的分类。
决策树的每一个内部节点是样本的某一个特征,
从这个节点出发的每一个弧表示这个特征的可能取值,
如果指向另一个内部节点,则表示需要对另外一个样本特征进行决策,
如果指向叶子结点,则表示已经得到了对样本分类的结论。

与一个训练集不矛盾的决策树可能存在多个,也可能不存在,
因此需要定义评价函数来衡量决策树的性能。
常用的评价函数有信息增益(Information Gain)和基尼指数(Gini Impurity)等。
信息增益是指得知特征$X$的信息而是的类$Y$的信息的不确定性减少的程度。
特征$A$对训练集$D$的信息增益$g(D, A)$定义为集合$D$的经验熵$H(D)$
与特征$A$给定条件下$D$的经验条件熵$H(D|A)$之差,即

\begin{equation}
    g(D, A) = H(D) - H(D|A)
\end{equation}

训练集已知各个样本的分类,训练集的熵即这些分类的随机程度,
假设训练集样本共有$K$类,第$k$类的样本集合为$C_k$,则

\begin{equation}
    H(D) = -\sum_{k=1}^K\frac{|C_k|}{|D|}\mathrm{log}_2\frac{|C_k|}{|D|}
\end{equation}

如果根据特征$A$的取值来分类,可能会使得随机程度降低,
降低的程度反应了这个特征所包含的信息量。
假设$A$共有$n$种可能的取值,第$i$种取值的样本的子集为$D_i$,
再假设子集$D_i$中属于第$k$类的样本的集合为$D_{ik}$,则

\begin{equation}
    \begin{split}
        H(D|A) &= \sum_{i=1}^n\frac{|D_i|}{|D|}H(D_i) \\
               &= -\sum_{i=1}^n\frac{|D_i|}{|D|}\sum_{k=1}^K\frac{|D_{ik}|}{|D_i|}\mathrm{log}_2\frac{|D_{ik}|}{|D_i|}
    \end{split}
\end{equation}

已知信息增益,即可使用ID3和C4.5算法构建决策树\cite{statistics}。

为了避免决策树过于复杂导致过拟合,可以将决策树叶节点的个数加入损失作为正则项。

\begin{equation}
    \mathrm{loss} = \sum_{t=1}^{|T|}N_tH_t(T)+\alpha|T|
\end{equation}

其中$H_t(T)$表示第t个叶节点上的经验熵,$|T|$为叶节点的个数,$\alpha$为参数。

本文使用Scikit Learn库中的\lstinline[basicstyle=\ttfamily]|DecisionTreeClassifier|进行训练及预测\cite{scikit}。
实验发现,使用\lstinline[basicstyle=\ttfamily]|max_leaf_nodes|作为正则
可以显著提高测试集性能,但是再叠加AdaBoost则会降低训练集和测试集性能。
AdaBoost 详见后续章节。


\section{结果}

\begin{table}[h]
    \centering
    \begin{tabular}{ccc}
        \hline
        模型& 训练集loss& 测试集 loss \\
        \hline
        决策树& 0.238& 27.653 \\
        决策树(叶节点39个)& 2.538& 2.544 \\
        决策树(叶节点39个,Ada)& 3.372& 3.378 \\
        Gradient Boost& & 2.46332 \\
        \hline
    \end{tabular}
\end{table}

\bibliography{main}{}
\bibliographystyle{plain}

\end{document}

